In this chapter, an introduction to required background theory for this thesis is given.
The work presented relies on \textbf{probabilistic robotics} and \textbf{computer vision} techniques.
A clear vocabulary and corresponding mathematical nomenclature will be established that will be used consistently in the remainder of this thesis.

\section{Probabilistic robotics}
Sensor are limited in what they can percieve (e.g., the range and resolution is subject to physical limitations).
Sensor are also subject to noise, which deviates sensor measurements in unpredictable ways.
This noise limits the information that can be extraced from the sensor measurements.
Robot actuators (e.g., motors) are also subject to noise, which introduces uncertainty.
Another source of uncertainty is caused by the robot's software, which uses approximate models of the world.
Model errors are a source of uncertainty that has often been ignored in robots.
Robots are real-time systems, limiting the amount of computation that can be done.
This requires the use of algorithmic approximations, but increases the amount of uncertainty even more.

For some robotic applications (e.g., assembly lines with controlled environment), uncertainty is a marginal factor.
However, robots operating in uncontrolled environments (e.g., homes, other planets) will have to cope with significant uncertainty.
Since robots are increasingly deployed in the open world, the issue of uncertainty has become a major challenge for designing capable robots.
\textit{Probabilistic robots} is a relatively new apprach to robotics that pays attentation to the uncertainty in perception and action.
Uncertainty is represented explicitly using the calculus of probability theory.
This means that probabilistic algorithms represent information by probability distributions over a whole space of guesses.
This allows to represent ambiguity and degree of belief in a mathematically sound way.

		\subsection{Motion}
		\subsection{Vision}
		\subsection{Localization}
		\subsection{Solution techniques}

		%\subsection{Extended Kalman filter}
		%\subsection{Particle filter}
		%\subsection{Information filter}
	\section{Computer vision}
		\subsection{Camera models}
			\subsubsection{Camera matrix}
		\subsection{Projective transformations}
			\subsubsection{Feature detection}
				%\subsubsection{SIFT}
				%\subsubsection{SURF}
		\subsection{Image stitching}
			%\subsubsection{Feature matching}
			%\subsubsection{Transformations}
			%	\subsubsection{Outlier detection}